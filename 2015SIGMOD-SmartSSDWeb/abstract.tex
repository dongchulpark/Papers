
\begin{abstract}
Recently, there is a renewed interest of \emph{In-Storage Computing} (ISC) in the context of solid state drives (SSDs), called ``Smart SSDs''. Unlike the traditional CPU-centric computing systems, ISC devices play a major role in computation by offloading key functions of host systems into the ISC devices, to take advantage of the high internal bandwidth, low I/O latency and computing capabilities.
It is challenging to determine what functions should be executed in the ISC devices.

%This work explores how to apply Smart SSDs to the operations used by web search engines.
%The major research issue is to determine which query processing steps of a search engine can be cost-effectively answered by Smart SSDs.
%To answer this question, we modified Apache Lucene (a popular open-source search engine) to utilize the Smart SSD of vendor X (X-SSD). We identified five commonly used operations in Lucene (and any search engine) that could potentially benefit from Smart SSDs and we codesigned their operation (with the collaboration of vendor X) between the host system and the X-SSD device. The five operations are \textsf{intersection}, \textsf{ranked intersection}, \textsf{ranked union}, \textsf{difference}, and \textsf{ranked difference}.
%Finally, we conducted extensive experiments to evaluate the performance and tradeoffs by using both synthetic datasets and real datasets (provided by a commercial large-scale search engine company). The experimental results show that, Smart SSD can generally reduce the query latency by a factor of 2-3$\times$ and energy consumption by 8-10$\times$ \textcolor{red}{(not precise)}.


This work explores how to apply Smart SSDs to Apache Lucene (a popular open-source search engine).
The major research issue is to determine which query processing steps of Lucene can be cost-effectively offloaded to Smart SSDs.
To answer this question, we identified five commonly used operations in Lucene (and any search engine) that could potentially benefit from Smart SSDs and we codesigned their operation (with the collaboration of an SSD vendor X) between the host system and the X-SSD device. The five operations are \textsf{intersection}, \textsf{ranked intersection}, \textsf{ranked union}, \textsf{difference}, and \textsf{ranked difference}.
Finally, we conducted extensive experiments to evaluate the performance and tradeoffs by using both synthetic datasets and real datasets (provided by a commercial large-scale search engine company). The experimental results show that, for some operations, Smart SSDs can reduce the query latency by a factor of 2-3$\times$ and energy consumption by 8-10$\times$.
\end{abstract} 