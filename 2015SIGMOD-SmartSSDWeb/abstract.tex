\begin{abstract}
Recently, there is a renewed interest of \emph{In-Storage Computing} (ISC) in the context of solid state drives (SSDs), called ``Smart SSDs". Unlike the traditional CPU-centric computing systems, it enables ISC devices to play a major role in computation by offloading key functions of host systems into ISC devices, to take the advantage of the high internal bandwidth, low I/O latency and computing capabilities.
It is challenging to determine what functions to be executed in the ISC devices. Existing works studied the query offloading in database systems to ISC devices.

This work explores how to apply Smart SSD to web search engine area, the first work in this area.
The major research issue is to determine what query processing steps in web search engines that could be cost-effectively answered by Smart SSD.
To answer this question, we identify five commonly used operations in web search engines that could be potentially benefit from Smart SSD: \textsf{intersection}, \textsf{ranked intersection}, \textsf{ranked union}, \textsf{difference}, and \textsf{ranked difference} and discuss the opportunities of the query offloading.
We closely work with an SSD vendor to implement these operations to Smart SSD.
We also integrate Smart SSD with an open-source search engine -- Apache Lucene.
Finally, we conduct extensive experiments to evaluate the performance and tradeoffs by using both synthetic datasets and real datasets (provided by a commercial large-scale search engine company). The experimental results show that, Smart SSD can generally reduce the query latency by a factor of 2-3$\times$ and energy consumption by 8-10$\times$.

\end{abstract} 