
\section{Related Work}\label{sec:relatedWork}
The idea of offloading computation to storage device (i.e., in-storage computing) has been around for decades. Many research efforts (both hardware and software sides) were dedicated to make it practical.

\textbf{Early works on in-storage computing.} As early as 1970s, some initial works propose to leverage specialized hardware (e.g., processor-per-track and processor-per-head) to improve query processing in storage devices (ie.., hard disks at that time).
For example, CASSM~\cite{Su1975} and RAP~\cite{Ozkarahan1977} follow the processor-per-track architecture to embed each track a processor.
The Ohio State Data Base Computer (DBC)~\cite{Kannan1978} and SURE~\cite{LeilichSZ78} follow the processor-per-head architecture to associate processing logic with each read/write head of a hard disk.
However, none of the systems turned out to be successful due to high design complexity and manufacturing cost.

\textbf{Later works on HDD in-storage computing.}
In late 1990s, when the bandwidth of hard disks continue to grow while the cost of powerful processors continue to drop, making it feasible to offload bulk computation to each individual disk. Researchers examine in-storage computing in terms of hard disks, e.g., active disk~\cite{Acharya1998ADP} or intelligent disk~\cite{Keeton1998}. The goal is to offload application-specific query operators inside hard disk, to save data movement. They examine active disk in database area, by offloading several primitive database operators, e.g., selection, group-by, sort. Later on, Erik et. al extends the application to data mining and multimedia area~\cite{Riedel1998ASL}. E.g., frequent sets mining, and edge detection. Although interesting, few real systems adopted the proposals, due to various reasons, including, limited hard disk bandwidth, computing power, and performance gains.




\textbf{Recent works on SSD in-storage computing.}
Recently, with the advent of SSD, which is a potential to replace HDD. People start to rethink about in-storage computing in the context of SSD, i.e., Smart SSD.
SSD offers many advantages over HDD, e.g., very high internal bandwidth and high computing power (because of the ARM processor techniques).
More importantly, executing code inside SSD can save a lot of energy because of less data movement and power-efficient embedded ARM processors. And, energy is becoming very critical today.
This, makes the concept of in-storage computing on SSD much more promising this time.
Industries like IBM started to install active SSD to the Blue Gene supercomputer to leverage the high internal bandwidth of SSD~\cite{Julich13}.
In this way, computing power and storage device are closely integrated. Teradata's Extreme Performance Appliance~\cite{Teradata20} is also an example of combining SSD and database functionalities together. Another example is Oracle's Exadata~\cite{Oracle2010}, which also started to offload complex processing into their storage servers.

SSD in-storage computing (or Smart SSD) is also attractive in academia. In database area, Kim et. al investigated pushing down the database scan operator to SSD~\cite{KimOPCL11}.
That work is based on simulation. Later on, Jaeyoung et. al~\cite{Do2013QPS} built a Smart SSD prototype on real SSDs. They integrated Smart SSD with Microsoft SQL Server to by offloading two operators: scan and aggregation. Woods et. al also built a prototype of Smart SSD with FPGAs~\cite{WoodsIA14}. Their target is also for database systems, but with more operators, e.g., group-by. They integrated the prototype with MySQL storage engine such as MyISAM and INNODB. In data mining area, Bae et. al investigated offloading functionalities like k-means and Aprior to Smart SSD~\cite{BaeKKOP13}. In data analytics area, De et. al propose to push down hash tables inside SSD~\cite{De2013}. There are also some work on offloading sorting~\cite{Young14}.

Unlike existing works, our work investigate the potential benefit of Smart SSD on web search area. To the best of our knowledge, this is the first work in this area. We built a Smart SSD on real commercial SSD products. We also integrated with an open source search engine -- Lucene.
