
\section{Related Work}\label{sec:relatedWork}
The idea of offloading computation to storage device (i.e., In-Storage Computing) has been around for decades. Many research efforts (both hardware and software sides) have been made to make it practical.

\textbf{Early work on In-Storage Computing.} As early as 1970s, some pieces of initial work have been proposed to leverage specialized hardware (e.g., processor-per-track and processor-per-head) for improving query processing in storage devices (i.e., hard disks at that time). For example, CASSM~\cite{Su1975} and RAP~\cite{Ozkarahan1977} followed the processor-per-track architecture to embed a processor per each track. The Ohio State Data Base Computer (DBC)~\cite{Kannan1978} and SURE~\cite{LeilichSZ78} followed the processor-per-head architecture to associate processing logic with each read/write head of a hard disk. However, none of the systems turned out to be successful due to high design complexity and manufacturing cost.

\textbf{Later work on HDD In-Storage Computing.}
In late 1990s, the bandwidth of hard disks kept growing while the cost of powerful processors kept dropping, which makes it feasible to offload bulk computation to each individual disk. Researchers started to explore in-storage computing in terms of hard disks (e.g., active disk~\cite{Acharya1998ADP} or intelligent disk~\cite{Keeton1998}). Their goal is to offload application-specific query operators inside hard disk in order to save data movement. They examined active disk in database area by offloading several primitive database operators (e.g., selection, group-by, sort). Later on, Erik et. al extended the application to data mining and multimedia area~\cite{Riedel1998ASL} (e.g., frequent sets mining and edge detection). Although interesting, few real systems adopted the proposals due to various reasons including limited hard disk bandwidth, computing power, and performance gains.



\textbf{Recent work on SSD In-Storage Computing.}
Recently, with the advent of SSD, people start to rethink about in-storage computing in the context of SSD (i.e., Smart SSD).
SSD offers many advantages over HDD such as very high internal bandwidth and high computing power. More importantly, executing codes inside SSD can save a lot of energy due to less data movement and power-efficient embedded ARM processors. %And, energy is becoming very critical today.
This makes the concept of in-storage computing on SSD much more practical and promising this time.
Industries like IBM started to install active SSD to their Blue Gene supercomputer to leverage the high internal bandwidth of SSD~\cite{Julich13}. In this way, computing power and storage device are closely integrated. Teradata's Extreme Performance Appliance~\cite{Teradata20} is another example of combining SSD and database functionality together. Oracle's Exadata~\cite{Oracle2010} also started to offload complex processing into their storage servers.

SSD in-storage computing (or Smart SSD) attracts academia as well. In database area, Kim et. al investigated pushing down the database scan operator to SSD~\cite{KimOPCL11}. That work is based on simulation. Later, Do et. al~\cite{Do2013QPS} built a Smart SSD prototype on real SSDs. They integrated Smart SSD with Microsoft SQL Server by offloading two operators: scan and aggregation. Woods et. al built another types of Smart SSD prototype with FPGAs~\cite{WoodsIA14}. Although they also targeted at database systems, they provided more operators such as group-by. They integrated the prototype with MySQL storage engine such as MyISAM and INNODB. In data mining area, Bae et. al studied offloading functions like k-means~\cite{BaeKKOP13}. In data analytic area, De et. al proposed to push down hash tables inside SSD~\cite{De2013}. There is also another study on offloading sorting~\cite{Young14}.

Unlike existing work, our work thoroughly investigates the potential benefit of Smart SSD on search engine area. To the best of our knowledge, this is the first study in this area.
